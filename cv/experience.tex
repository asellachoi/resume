%!TEX root = ../cv.tex
% -*- root: ../cv.tex -*-

%-------------------------------------------------------------------------------
%	SECTION TITLE
%-------------------------------------------------------------------------------
\cvsection{Experience}


%-------------------------------------------------------------------------------
%	CONTENT
%-------------------------------------------------------------------------------
\begin{cventries}

%---------------------------------------------------------
  \cventry
    {Computer Vision based Driver Status Monitoring System} % Job title
    {LG Electronics AI Research Institute} % Organization
    {Senior Research Engineer} % Location
    {Mar. 2015 - Current} % Date(s)
    {
      \begin{cvitems} % Description(s) of tasks/responsibilities
        \item {Detection and localization for steering wheel and spoke region}
        \item {Light-weight and reliable computer vision algorithm for embedded vehicle system}
      \end{cvitems}
    }

%---------------------------------------------------------
  \cventry
%   Jan.14 –  Feb. 15   PM, KIST
%       “시각 기반 지도 인식 모듈 개발”
% - 모바일 기기에서 특징점 매칭 방법을 이용하여 시각 인식 기반의 증강현실 및 영상 인식 엔진을 개발
    {Computer Vision based Map Recognition} % Job title
    {KIST (Korea Institute of Science and Technology) Research Project} % Organization
    {Project Leader} % Location
    {Jan. 2014 - Feb. 2015} % Date(s)
    {
      \begin{cvitems} % Description(s) of tasks/responsibilities
        \item {Local feature based augmented reality and image recognition engine for mobile computing}
        \item {Develop a feature point filtering algorithm for high performance and speed}
        \item {Design multi-threaded image matching and tracking system for multiple objects}
        \item {Supply mobile augmented reality engine to develop Seoul Tour Guide application}
      \end{cvitems}
    }

%---------------------------------------------------------
  \cventry
%   Jan.13 -  Nov. 13   연구원, LG전자 전자기술원 산학과제
% “360° 프로젝션 및 인터랙션 기술개발”
% - 360° 회전가능한 프로젝터를 이용한 퍼베이시브 컴퓨팅 환경 개발과 이를 위한 Pan-Tilt에 따른 프로젝션 정규화 기술 개발 및 터치 기반 상호작용 기술 개발
    {$360^{\circ}$ Around Projection Interface} % Job title
    {LG Electronics Advanced Research Institute Project} % Organization
    {Researcher} % Location
    {Jan. 2013 - Nov. 2013} % Date(s)
    {
      \begin{cvitems} % Description(s) of tasks/responsibilities
        \item {Develop a pervasive computing environment using $360^{\circ}$ pan-tilt steering projector}
        \item {Research a computer vision based projecting image rectification algorithm for pan-tilt projector}
        \item {Develop a projector-camera calibration algorithm using depth camera}
      \end{cvitems} 
    }

%---------------------------------------------------------
  \cventry
%   May.12 -  Apr. 15   PM, 연구재단 중견연구자지원사업
% “Smart space 공간 제어를 위한 Natural User Interface 기술 연구”
% - 다양한 NUI 기술들이 혼재한 Smart Space에서 이러한 기기들 사이의 연결성을 제공하고, 복합적 상호작용을 제공하기 위한 Interaction Framework 개발 (학위논문)
% - 센서와 비전을 혼용한 사용자 의도 파악 기술 연구 및 Smart Space 환경 제어 기술 개발
    {Research on Natural User Interface to Control Smart Space Environment} % Job title
    {National Research Foundation Senior Research Project} % Organization
    {Project Leader} % Location
    {May. 2012 - Apr. 2015} % Date(s)
    {
      \begin{cvitems} % Description(s) of tasks/responsibilities
        \item {Research on user recognition using embedded sensors and computer vision hybrid}
        \item {Research on an interaction framework which supports various natural user interaction technologies}
        \item {Develop UX scenarios for pervasive AR environment}
      \end{cvitems}
    }

%---------------------------------------------------------
  \cventry
%   Aug.11 - Aug.12   PM, 삼성SDS
% “고성능 Vision 기반 Motion 인식 기술”
% - 손 동작 인식 엔진 개발 및 이를 활용한 응용 프로그램 개발
% - 다양한 조명 환경에서의 손 분리 기술 개발
% - 손의 포스처와 제스처 인식 기술 연구
    {Research on Computer Vision based Motion Recognition} % Job title
    {Samsung SDS Research Project} % Organization
    {Project Leader} % Location
    {Aug. 2011 - Aug. 2012} % Date(s)
    {
      \begin{cvitems} % Description(s) of tasks/responsibilities
        \item {Develop hands motion recognition algorithm using depth image}
        \item {Develop natural user interaction applications using hands postures and gestures}
        \item {Research on hand segmentation algorithm to overcome lighting effects}
        \item {Recognize hands posture and gesture}
      \end{cvitems}
    }

%---------------------------------------------------------
  \cventry
%   Apr.11 - Nov.11   AR Team Leader, 삼성전자 DMC 연구소, 
% "비전 기능의 CPU/GPU 혼합 가속 기법  연구 및 검증”
% - GPU 기반 병렬처리를 위한 Markerless 증강현실 가속화 기술 연구
    {Accelerate computer vision algorithm  using CPU/GPU parallelism} % Job title
    {Samsung Electronics Research Project} % Organization
    {AR Team Leader} % Location
    {Apr. 2011 - Nov. 2011} % Date(s)
    {
      \begin{cvitems} % Description(s) of tasks/responsibilities
        \item {Develop mobile augmented reality algorithm}
        \item {Research on acceleration method using CPU/GPU parallelism}
        \item {Design multi-threaded AR system to support detection and tracking simultaneously}
      \end{cvitems}
    }

%---------------------------------------------------------
  \cventry
% Mar.09 - Dec.10   PM, (주)아코엔터테인먼트,
% "증강현실 및 멀티터치 요소기술개발"
% - 마커 및 마커리스 증강현실 엔진 개발 및 요소기술 개선
% - 교육 환경에서의 증강현실 응용 어플리케이션 개발
    {Research and develop educational augmented reality engine} % Job title
    {AKO Entertainment Research Project} % Organization
    {Project Leader} % Location
    {Mar. 2009 - Dec. 2010} % Date(s)
    {
      \begin{cvitems} % Description(s) of tasks/responsibilities
        \item {Develop marker based augmented reality engine}
        \item {Design educational fiducial marker and develop its recognition system}
        \item {Develop educational AR contents and applications}
      \end{cvitems}
    }

%---------------------------------------------------------
  \cventry
% Sep.08 - Aug.11   PM, 한국과학재단 특정기초연구,
% "사용자 참여형 교육용 e-Book 설계 및 제작 기술"
% - 증강현실의 교육적 적용에 관한 연구 및 교육 관련 증강현실 Application 개발
% - 증강현실 저작 도구 개발
    {Design and develop an augmented reality based e-Book for education} % Job title
    {Korea Science and Engineering Foundation} % Organization
    {AR Team Leader} % Location
    {Sep. 2008 - Aug. 2011} % Date(s)
    {
      \begin{cvitems} % Description(s) of tasks/responsibilities
        \item {Develop augmented reality contents to enhance educational performance}
        \item {Develop augmented reality authoring tools for ordinary teachers}
      \end{cvitems}
    }

%---------------------------------------------------------
  \cventry
% Sep.08 - Jun.09   PL, 삼성종합기술원,
% "Perceptive Display 상에서의 Object Recognition을 위한 Visual/IR Tag 기술 연구"
% - 적외선 영역에서의 비가시 태그(Invisible Tag) 인식 기술 연구
% - 대형 Display 환경에서의 Tag 기반 상호작용 기술 연구
    {Research on Visual/IR Tags for Perceptive Display Environment} % Job title
    {SAIT (Samsung Advanced Institute of Technology) Research Project} % Organization
    {Researcher} % Location
    {Sep. 2008 - Aug. 2011} % Date(s)
    {
      \begin{cvitems} % Description(s) of tasks/responsibilities
        \item {Research on detecting invisible fiducial tags using infra-red frequency}
        \item {Design interaction between large display and mobile device using fiducial marker}
      \end{cvitems}
    }

%---------------------------------------------------------
  \cventry
% Mar.06 – Dec.08   연구원, 한국과학재단 특정기초연구,
% "차세대 PC환경 지원을 위한 영상인식 및 표현 기술에 대한 연구"
% - Color 영상처리를 통한 마커 인식 기술 연구
% - 마커 기반 증강현실 기술 연구
    {Research of Image Recognition and Visualization for Supporting Next Generation PC Environment} % Job title
    {Korea Science and Engineering Foundation} % Organization
    {Researcher} % Location
    {Mar. 2006 - Dec. 2008} % Date(s)
    {
      \begin{cvitems} % Description(s) of tasks/responsibilities
        \item {Research on detecting color-based fiducial marker}
        \item {Research on color marker-based augmented reality}
      \end{cvitems}
    }
%---------------------------------------------------------
\end{cventries}
